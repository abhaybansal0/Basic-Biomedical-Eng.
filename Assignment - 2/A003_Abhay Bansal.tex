\documentclass[12pt]{article}
\usepackage{graphicx}
\graphicspath{{images}}
\usepackage{hyperref}
\hypersetup{colorlinks=true,citecolor=blue,linkcolor=blue}

\usepackage{pdfpages}




\begin{document}
\includepdf[pages=1,fitpaper]{Frountpage2}



\tableofcontents
\newpage


\section {\textsl{\textbf{INTRODUCTION}}}

As hospitals, physician groups and health systems continue to consolidate, navigating a sustainable route to clinical and operational alignment is more complex. Things won’t be slowing down for the HIM industry anytime soon. “The volume of data healthcare organizations are now collecting pales in comparison to the amount of data that will be generated in a year,” says Anatoly Geyfman, CEO of healthcare analytics company Carevoyance. \newline\newline
 This expanding branch of the healthcare field could be your ticket to the job stability you’ve always dreamed of.This overview of health information management history will give you all the facts you need about this growing industry. With a varied history and a promising future, HIM could be your new career.





\section{Health Records}


Documentation became wildly popular and was used throughout the nation after healthcare providers realized that they were better able to treat patients with complete and accurate medical history. Health records were soon recognized as being critical to the safety and quality of the patient experience.\newline\newline
 The ACOS standardized these clinical records by establishing the American Association of Record Librarians, a professional association that exists today under the name American Health Information Management Association (AHIMA). These early medical records were documented on paper, which explains the name “record librarians.”

\section{Millennial medical records}

A wave of medical errors and patient deaths caused by healthcare providers renewed the search for a viable EHR system in 2000. Electronic health records would allow “providers to make better decisions and provide better care” while “reducing the incidence of medical error by improving the accuracy and clarity of medical records.”


\section{The Future of Health Care System}
Hospitals as a physical entity will change — they’ll look different. Sure, there will be more care delivery in a patient’s home. But there will also be care outside the home: the corner pharmacy taking on a bigger role or assisted living. Another avenue is self-service point of care — think of Amazon delivering at-home Covid-19 test kits directly to consumers.\newline\newline 
 We also need to embrace new ways to consume healthcare and payers need to incentivize and compensate healthcare providers to leverage technology. As Covid-19 reduced or eliminated elective procedures, routine visits and other bread-and-butter revenue streams have been slashed.
 
 \section{Research And Development}
 
In review of the research literature, the committee learned that there is ample foundational knowledge to apply a human factors lens to home health care, particularly as improvements are considered to make health care safe and effective in the home. However, much of what is known is not being translated effectively into practice, neither in design of equipment and information technology or in the effective targeting and provision of services to all those in need. \newline
Consequently, the four recommendations that follow support research and development to address knowledge and communication gaps and facilitate provision of high-quality health care in the home.
 
 
\section {\textsl{\textbf{CONCLUSION}}} 
 
As knowledge increases about the genetic bases of disease, the healthcare system will make greater use of gene therapies, developing ways to prevent genetically caused diseases. The interaction of managed care and chronic disease is a complex nexus that requires new research paradigms, which should be as integrative as possible. Organizational changes in health care are more likely to succeed when health care professionals have the change, and recognize the value the value of the change, including perceiving the benefit of the change for patients.\newpage
The growth of both managed care and chronic disease have cast work force issues into bold relief, demanding reanalysis of the optimal roles of generalists and subspecialists.Chronic disease is responsible for a large and growing proportion of health care utilization in the United States today, but those suffering from these diseases are also highly heterogeneous.
 \newline\newline\newline\newline\newline\newline\newline\newline\newline\newline\newline
\begin{center}
\textbf{\textsl{\textsl{Thank You}}}
\end{center}







\end{document}