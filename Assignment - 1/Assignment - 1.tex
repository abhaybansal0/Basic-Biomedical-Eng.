\documentclass[12pt]{article}
\usepackage{graphicx}
\graphicspath{{images}}
\usepackage{hyperref}
\hypersetup{colorlinks=true,citecolor=blue,linkcolor=blue}

\usepackage{pdfpages}




\begin{document}
\includepdf[pages=1,fitpaper]{Frountpage2}



\tableofcontents


\section {\textsl{PACEMAKER}}


A cardiac pacemakeris a medical device that generates electrical impulses delivered by electrodes to cause the heart muscle chambers to contract and therefore pump blood; by doing so this device replaces and/or regulates the function of the electrical conduction system of the heart.
\newline\newline
The primary purpose of a pacemaker is to maintain an adequate heart rate, either because the heart's natural pacemaker is not fast enough, or because there is a block in the heart's electrical conduction system.

\begin{figure}
\centering
\includegraphics[scale=0.8]{Pacemaker(1)}
\caption{Pacemaker}
\end{figure}

\subsection{Methods of pacing}
\begin{itemize}
\item Percussive pacing
\item Transcutaneous pacing
\item Epicardial pacing
\item Transvenous pacing
\item Permanent transvenous pacing
\item Leadless pacing
\end{itemize}

\subsection{Basic function}

Modern pacemakers usually have multiple functions. The most basic form monitors the heart's native electrical rhythm. When the pacemaker wire or "lead" does not detect heart electrical activity in the chamber atrium or ventricle within a normal beat-toeat time period most commonly one second it will stimulate either the atrium or the ventricle with a short low voltage pulse.


\subsection{Biventricular pacing}

Cardiac resynchronization therapy (CRT) is used for people with heart failure in whom the left and right ventricles do not contract simultaneously, which occurs in approximately $25–50\%$ of heart failure patients.

\subsection{His bundle pacing}

Conventional placement of ventricular leads in or around the tip or apex of the right ventricle, or RV apical pacing, can have negative effects on heart function. It has been associated with increased risk of atrial fibrillation, heart failure, weakening of the heart muscle and potentially shorter life expectancy.

\subsection{Advancements in function}

A major step forward in pacemaker function has been to attempt to mimic nature by utilizing various inputs to produce a rate-responsive pacemaker using parameters such as the QT interval.

\subsection{Considerations}
\begin{itemize}
\item Insertion
\item Periodic pacemaker checkups
\item Magnetic fields, MRIs, and other lifestyle issues
\item Turning off the pacemaker
\item Complications
\end{itemize}

\section {\textsl{GAMMA CAMERA}}
\begin{figure}
\centering
\includegraphics[scale=0.8]{GammaScanner}
\caption{Gamma camera}

\end{figure}
A gamma camera, also called a scintillation camera or Anger camera, is a device used to image gamma radiation emitting radioisotopes, a technique known as scintigraphy.\newline\newline
The applications of scintigraphy include early drug development and nuclear medical imaging to view and analyse images of the human body or the distribution of medically injected, inhaled, or ingested radionuclides emitting gamma rays.

\subsection{Imaging techniques}
Single Photon Emission Computed Tomography imaging, as used in nuclear cardiac stress testing, is performed using gamma cameras. Usually one, two or three detectors or heads, are slowly rotated around the patient's torso.\newline 
Multi-headed gamma cameras can also be used for positron emission tomography (PET) scanning, provided that their hardware and software can be configured to detect "coincidences" 


\subsection{Construction}
A gamma camera consists of one or more flat crystal planes (or detectors) optically coupled to an array of photomultiplier tubes in an assembly known as a "head", mounted on a gantry. The gantry is connected to a computer system that both controls the operation of the camera and acquires and stores images.\newline
The system accumulates events, or counts, of gamma photons that are absorbed by the crystal in the camera. Usually a large flat crystal of sodium iodide with thallium doping in a light-sealed housing is used.

\subsection{Signal processing}
Hal Anger developed the first gamma camera in 1957. His original design, frequently called the Anger camera, is still widely used today. The Anger camera uses sets of vacuum tube photomultipliers (PMT). The electronic circuit connecting the photodetectors is wired so as to reflect the relative coincidence of light fluorescence as sensed by the members of the hexagon detector array.
\newline\newline
The location of the interaction between the gamma ray and the crystal can be determined by processing the voltage signals from the photomultipliers; in simple terms, the location can be found by weighting the position of each photomultiplier tube by the strength of its signal, and then calculating a mean position from the weighted positions.

\subsection{Spatial resolution}

In order to obtain spatial information about the gamma-ray emissions from an imaging subject  a method of correlating the detected photons with their point of origin is required.\newline
The conventional method is to place a collimator over the detection crystal/PMT array.The individual holes limit photons which can be detected by the crystal to a cone shape; the point of the cone is at the midline center of any given hole and extends from the collimator surface outward.

\newpage
\section {\textsl{NEBULIZER}}
\begin{figure}
\centering
\includegraphics[scale=1.2]{Nebulizer(1)}
\caption{Nebulizer}
\end{figure}

In medicine, a nebulizer is a drug delivery device used to administer medication in the form of a mist inhaled into the lungs. Nebulizers are commonly used for the treatment of asthma, cystic fibrosis, COPD and other respiratory diseases or disorders.\newline\newline
They use oxygen, compressed air or ultrasonic power to break up solutions and suspensions into small aerosol droplets that are inhaled from the mouthpiece of the device. An aerosol is a mixture of gas and solid or liquid particles.

\subsection{USES}
\begin{itemize}
\item \textsl{GUIDELINES}\newline
Various asthma guidelines, such as the Global Initiative for Asthma Guidelines [GINA], and United States Guidelines for Diagnosis and Treatment of Asthma each recommend metered dose inhalers in place of nebulizer-delivered therapies.
\item \textsl{EFFECTIVENESS}\newline
Recent evidence shows that nebulizers are no more effective than metered-dose inhalers (MDIs) with spacers.[5] An MDI with a spacer may offer advantages to children who have acute asthma.


\end{itemize}

\subsection{Aerosol deposition}
The lung deposition characteristics and efficacy of an aerosol depend largely on the particle or droplet size. Generally, the smaller the particle the greater its chance of peripheral penetration and retention. However, for very fine particles below $0.5  \mu$m in diameter there is a chance of avoiding deposition altogether and being exhaled.

\subsection{Types of nebulizers}

\begin{itemize}
\item \textsl{Pneumatic}
	\begin{itemize}
	\item Jet nebulizer
	\end{itemize}
\item \textsl{Mechanical}
	\begin{itemize}
	\item Soft mist inhaler
	\end{itemize}
\item \textsl{Electrical}
	\begin{itemize}
	\item Ultrasonic wave nebulizer
	\item Vibrating mesh technology
	\end{itemize}
\end{itemize}

\subsection{Use and attachments}
Nebulizers accept their medicine in the form of a liquid solution, which is often loaded into the device upon use. Corticosteroids and bronchodilators such as salbutamol (albuterol USAN) are often used, and sometimes in combination with ipratropium.
\newline

The reason these pharmaceuticals are inhaled instead of ingested is in order to target their effect to the respiratory tract, which speeds onset of action of the medicine and reduces side effects, compared to other alternative intake routes.
\newline

 
\section {\textsl{PULSE OXIMETER}}
\begin{figure}
\centering
\includegraphics[scale=1]{Oximeter}
\caption{Pulse Oximeter}
\end{figure}

A pulse oximeter is a medical device that indirectly monitors the oxygen saturation of a patient's blood (as opposed to measuring oxygen saturation directly through a blood sample) and changes in blood volume in the skin, producing a photoplethysmogram that may be further processed into other measurements.
\newline\newline
The pulse oximeter may be incorporated into a multiparameter patient monitor. Most monitors also display the pulse rate. Portable, battery-operated pulse oximeters are also available for transport or home blood-oxygen monitoring.


 \subsection{Advantages} 
 \begin{itemize}
 \item Pulse oximetry is particularly convenient for noninvasive continuous measurement of blood oxygen saturation.
 \item  Pulse oximetry is useful in any setting where a patient's oxygenation is unstable, recovery, need for supplemental oxygen.
 \item Because of their simplicity of use and the ability to provide continuous oxygen saturation values, pulse oximeters are of critical importance in emergency medicine.
 \item Pulse oximeter are useful for patient with respiratory or cardiac problems.

\end{itemize}
 
 \subsection{Limitations}

\begin{itemize}
 \item A pulse oximeter cannot determine the metabolism of oxygen, or the amount of oxygen being used by a patient.
 \item Pulse oximetry solely measures hemoglobin saturation, not ventilation and is not a complete measure of respiratory sufficiency.
 \item Pulse oximetry technologies differ in their abilities to provide accurate data during conditions of motion and low perfusion.
\end{itemize}



 \subsection{Equipment}
 \begin{itemize}
 \item Some smart watches with activity tracking incorporate an oximeter function.
\item Mobile app pulse oximeters use the flashlight and the camera of the phone, instead of infrared light used in conventional pulse oximeters.

\end{itemize}

\subsection{Mechanism}
A blood-oxygen monitor displays the percentage of blood that is loaded with oxygen. More specifically, it measures what percentage of hemoglobin, the protein in blood that carries oxygen, is loaded.\newline\newline
One LED is red, with wavelength of $660$ nm, and the other is infrared with a wavelength of $940$ nm. Absorption of light at these wavelengths differs significantly between blood loaded with oxygen and blood lacking oxygen.

\subsection{Mode of operation}
A typical pulse oximeter uses an electronic processor and a pair of small light-emitting diodes (LEDs) facing a photodiode through a translucent part of the patient's body, usually a fingertip or an earlobe. \newline

One LED is red, with wavelength of $660$ nm, and the other is infrared with a wavelength of $940$ nm. Absorption of light at these wavelengths differs significantly between blood loaded with oxygen and blood lacking oxygen.

 \subsection{Derived measurements}
 Due to changes in blood volumes in the skin, a plethysmographic variation can be seen in the light signal received (transmittance) by the sensor on an oximeter. The variation can be described as a periodic function, which in turn can be split into a DC component (the peak value)[a] and an AC component.
\newline\newline



 \section{\textsl{PET SCANNER}}
Positron emission tomography (PET) is a functional imaging technique that uses radioactive substances known as radiotracers to visualize and measure changes in metabolic processes, and in other physioligical activities including blood flow, regional chemical composition, and absorption. \newline\newline
Different tracers are used for various imaging purposes, depending on the target process within the body.PET is a common imaging technique, a medical scintillogrophy technique used in nuclear medicine.


\subsection{USES}

PET is both a medical and research tool used in pre-clinical settings. it is heavily used in imaging of tumors and the search of metastases within the field of clinical oncology, and for cilinical diagnosis of certain diffuse brain diseases such as those causing various types of dementais.


\begin{itemize}
\item {\textbf{\textsl{Oncology}}}
	\newline PET scanning with the tracer 18F-FDG is widely used in clinical oncology. FDG is a glucose analog that is taken up by glucose-using cells and phosphorylated by hexokinase.
\item {\textbf{\textsl{Neuroimaging}}}
	\begin{itemize}
	\item Neurology
	\item Neuropsychology
	\item Psychiatry
	\item Stereotactic surgery and radiosurgery
	\end{itemize}
\item {\textbf{\textsl{Cardiology}}}
	\newline Cardiology, atherosclerosis and vascular disease study, can help in identifying hibernating myocardium.
	
\end{itemize}


\begin{figure}
\centering
\includegraphics[scale=0.95]{PETscanner}
\caption{PET Scanner}
\end{figure}


\subsection{Safety}
 The amount of radiation in 18F-FDG is similar to the effective dose of spending one year in the American city of Denver. Average civil aircrews are exposed to $3 mSv/year$, and the whole body occupational dose limit for nuclear energy workers in the USA is $50mSv/year$.
 
\subsection{Emission}
To conduct the scan, a short-lived radioactive tracer isotope is injected into the living subject (usually into blood circulation). \newline \newline
Each tracer atom has been chemically incorporated into a biologically active molecule. There is a waiting period while the active molecule becomes concentrated in tissues of interest; then the subject is placed in the imaging scanner.
\newline\newline\newline\newline\newline\newline\newline\newline\newline\newline\newline
\begin{center}
\textbf{\textsl{Thank You}}
\end{center}







\end{document}